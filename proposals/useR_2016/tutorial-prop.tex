\documentclass[]{article}
\usepackage{lmodern}
\usepackage{amssymb,amsmath}
\usepackage{ifxetex,ifluatex}
\usepackage{fixltx2e} % provides \textsubscript
\ifnum 0\ifxetex 1\fi\ifluatex 1\fi=0 % if pdftex
  \usepackage[T1]{fontenc}
  \usepackage[utf8]{inputenc}
\else % if luatex or xelatex
  \ifxetex
    \usepackage{mathspec}
    \usepackage{xltxtra,xunicode}
  \else
    \usepackage{fontspec}
  \fi
  \defaultfontfeatures{Mapping=tex-text,Scale=MatchLowercase}
  \newcommand{\euro}{€}
\fi
% use upquote if available, for straight quotes in verbatim environments
\IfFileExists{upquote.sty}{\usepackage{upquote}}{}
% use microtype if available
\IfFileExists{microtype.sty}{%
\usepackage{microtype}
\UseMicrotypeSet[protrusion]{basicmath} % disable protrusion for tt fonts
}{}
\usepackage[margin=1in]{geometry}
\ifxetex
  \usepackage[setpagesize=false, % page size defined by xetex
              unicode=false, % unicode breaks when used with xetex
              xetex]{hyperref}
\else
  \usepackage[unicode=true]{hyperref}
\fi
\hypersetup{breaklinks=true,
            bookmarks=true,
            pdfauthor={},
            pdftitle={useR! 2016 Tutorial Proposal},
            colorlinks=true,
            citecolor=blue,
            urlcolor=blue,
            linkcolor=magenta,
            pdfborder={0 0 0}}
\urlstyle{same}  % don't use monospace font for urls
\usepackage{longtable,booktabs}
\setlength{\parindent}{0pt}
\setlength{\parskip}{6pt plus 2pt minus 1pt}
\setlength{\emergencystretch}{3em}  % prevent overfull lines
\providecommand{\tightlist}{%
  \setlength{\itemsep}{0pt}\setlength{\parskip}{0pt}}
\setcounter{secnumdepth}{0}

%%% Use protect on footnotes to avoid problems with footnotes in titles
\let\rmarkdownfootnote\footnote%
\def\footnote{\protect\rmarkdownfootnote}

%%% Change title format to be more compact
\usepackage{titling}

% Create subtitle command for use in maketitle
\newcommand{\subtitle}[1]{
  \posttitle{
    \begin{center}\large#1\end{center}
    }
}

\setlength{\droptitle}{-2em}
  \title{useR! 2016 Tutorial Proposal}
  \pretitle{\vspace{\droptitle}\centering\huge}
  \posttitle{\par}
  \author{}
  \preauthor{}\postauthor{}
  \date{}
  \predate{}\postdate{}


% Redefines (sub)paragraphs to behave more like sections
\ifx\paragraph\undefined\else
\let\oldparagraph\paragraph
\renewcommand{\paragraph}[1]{\oldparagraph{#1}\mbox{}}
\fi
\ifx\subparagraph\undefined\else
\let\oldsubparagraph\subparagraph
\renewcommand{\subparagraph}[1]{\oldsubparagraph{#1}\mbox{}}
\fi

\begin{document}
\maketitle



\subsection{Tutorial Title}\label{tutorial-title}

Programming with models: An introduction to NIMBLE, a BUGS-compatible system for fitting and programming with hierarchical statistical models using MCMC and more
%Beyond the black box: Flexible programming of hierarchical modeling
%algorithms for
%BUGS-compatible models using NIMBLE

\subsection{Instructor Details}\label{instructor-details}

\begin{longtable}[c]{@{}ll@{}}
\toprule
Name: Chris Paciorek &\tabularnewline
&\tabularnewline
Institution: University of California, Berkeley&\tabularnewline
&\tabularnewline
Address: Department of Statistics, 367 Evans Hall, Berkeley CA 94720 &\tabularnewline
&\tabularnewline
Email: paciorek@stat.berkeley.edu &\tabularnewline
\bottomrule
\end{longtable}

\subsection{Short Instructor
Biography}\label{short-instructor-biography}

Christopher Paciorek is an associate research statistician, lecturer, and statistical computing consultant in the Department of Statistics at UC Berkeley. He has 15 years of experience in applications of Bayesian statistics with a wide variety of published analyses in environmental and public health applications (\url{http://www.stat.berkeley.edu/~paciorek/cv/cv.html}).He has extensive experience developing and presenting workshops on statistical computing topics, including intensive two-day bootcamps on R (\url{https://github.com/berkeley-scf/r-bootcamp-2015}) and  shorter workshops on parallel programming (\url{https://github.com/berkeley-scf/parallel-scf-2015}), using Spark and MapReduce (\url{https://github.com/berkeley-scf/spark-workshop-2014}), using GPUs (\url{https://github.com/berkeley-scf/gpu-workshop-2014}), and using C/C++ from R (\url{http://statistics.berkeley.edu/computing/cpp}). He has taught the graduate-level statistical computing class at UC Berkeley for five years (\url{https://github.com/berkeley-stat243/stat243-fall-2015}). He is a co-developer of the NIMBLE package (\url{http://R-nimble.org} and \url{http://arxiv.org/abs/1505.05093}) and the developer of the bigGP (\url{http://www.jstatsoft.org/v63/i10/}) and spectralGP packages (\url{http://www.jstatsoft.org/v19/a2}).  


\subsection{Brief Description of
Tutorial}\label{brief-description-of-tutorial}

This tutorial will introduce attendees to the NIMBLE system for programming with hierarchical models in R. The tutorial will first show how to specify a hierarchical statistical model using BUGS syntax and fit that model using MCMC. Attendees will learn how to customize the MCMC for better performance and how to specify one's own statistical distributions and functions for use within the BUGS syntax. Next the tutorial will demonstrate how to use other non-MCMC algorithms to fit and evaluate models. Finally, the tutorial will show how to write algorithms for use with hierarchical models. 

\subsection{Goals}\label{goals}

\begin{enumerate}
\def\labelenumi{\arabic{enumi}.}
\tightlist
\item
 Be able to specify hierarchical statistical models using BUGS
\item
 Be able to fit and assess models using NIMBLE's default MCMC system
\item
  Be able to customize the MCMC for a model
\item
  Be able to specify user-provided distributions and functions to extend BUGS
\item
Be able to use other algorithms such as MCEM and particle filtering
\item
Be able to code one's own algorithms using NIMBLE's flexible system for programming algorithms
\end{enumerate}

\subsection{Detailed Outline}\label{detailed-outline}


\begin{enumerate}
\item A complete worked example to illustrate the system
\item Specifying models with BUGS and manipulating models
\begin{enumerate}
\item Writing a model in BUGS
\item Creating a NIMBLE model
\item Working with the model in R
\end{enumerate}
\item Fitting models with MCMC
\begin{enumerate}
\item Building a default MCMC
\item Using the MCMC output
\item Customizing your MCMC: blocking parameters and choosing samplers
\end{enumerate}
\item Using other algorithms provided by NIMBLE
\item Writing your own algorithms using nimbleFunctions
\begin{enumerate}
\item A basic nimbleFunction for fast execution of R code
\item Using your own distributions and functions to extend BUGS
\item Writing an algorithm as a nimbleFunction
\end{enumerate}
\end{enumerate}

\subsection{Justification}\label{justification}

Fitting hierarchical models is at the heart of many statistical and machine learning workflows in a wide variety of fields, including statistics, machine learning/robotics/AI, biology (ecology, phylogenetics), social science (economics, education research, psychology, quantitative political science), public health (epidemiology, health services research, global health), and more. BUGS-based software tools (WinBUGS, JAGS) have been very popular but have limitations, specifically in only providing a user a black box MCMC. NIMBLE is much more flexible and opens up a variety of options for those fitting models and those hoping to provide algorithms that others can then use on their own models. 


\subsection{Background Knowledge}\label{background-knowledge}

Familiarity with R, hierarchical statistical models/Bayesian statistics, and basic understanding of Markov chain Monte Carlo (MCMC)

\subsection{Expected Number of
Attendees}\label{expected-number-of-attendees}

15-30

\end{document}
